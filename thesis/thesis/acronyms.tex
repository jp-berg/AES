% !TEX root = thesis.tex
% List of acronyms
%-- \ac{betai} f�gt die Abk�rzung ein. Beim ersten Vorkommen der Abk�rzung wird zun�chst der ganze Begriff angegeben und das Akronym bzw. die Abk�rzung in Klammern.
%-- \acs{betai} wird dann nur das Symbol bzw. die Abk�rzung im Text gedruckt
%-- \acf{betai} gibt zus�tzlich noch die Erkl�rung aus
%-- \acl{betai} f�gt lediglich die Beschreibung ein.
%\chapter{Acronyms}
\begin{acronym}
 \setlength{\itemsep}{0.2em}
 \acro{BPMN}{business process model and notation}
 \acro{FFT}{fast Fourier transformation}
 \acro{IMEI}{international mobile station equipment identity}
 \acro{MAC}{media access control}
 \acro{MEMS}{microelectromechanical system}
 \acro{PCA}{principal component analysis}
 \acro{RSS}{root sum square}
 \acro{UDID}{unique device identifier}

 \acro{AES}{Advanced Encryption Standard}
 \acro{WBS}{work breakdown structure}
 \acro{FPGA}{Field Programmable Gate Array}
 \acro{WP}{Work Packages}
 \acro{DES}{Data Encryption Standard}
 \acro{NIST}{National Institute of Standards and Technology}
 \acro{FIPS}{Federal Information Processing Standard}
 \acro{AES-NI}{Advanced Encryption Standard New Instructions}
 \acro{US}{United States}
 \acro{NSA}{National Security Agency}
\end{acronym}
