% !TEX root = thesis.tex
% Always start a chapter with a short but informative text about the following sections\@. Point out the relevance of the sections and create interconnections between them\@. Never ever just write a single sentence here\@. Furthermore,\ you are strongly advised to respect the hints given in this template\@.

In this chapter, we will give an overview on our motivation and the setting of the implementation of the Advanced Encryption Standard, followed by an outline of the structure of this thesis.
 

\section{Motivation}
\label{ch:motivation}
The more our lives shift into the internet the more important becomes protection of it and the data it represents, not least, because of the ever increasing dangers lurking in the world wide web. This protection can partially be archieved through encryption, which is why it and with that the Advanced Encryption Standard is found nearly everywhere in the digital realm. The majority  of instant messengers used \cite{instantmessages} are encrypted with the Signal protocol\cite{whatsapp}\cite{fbmessenger}\, which uses AES at its heart \cite[ch. 5.2]{signal}. Windows only enables TLS with AES as a default \cite{wintls} and nearly 85\% of webpages loaded via Firefox use https and thus TLS \cite{fftelem}. Android \cite{android} and iOS \cite{ios}encrypt all devices by default with AES. 

But how does this encryption algorithm work and why does it seem to enjoy such widespread trust? The following document gathers some information on AES and outlines the implementation process of the algorithm.

\section{Setting}
\label{ch:setting}

This document provides detailed information over the algorithm itself and the planning process involved in an implementation of AES, showing which parts were done by which team member and demonstrating project management ability. It is part of the project seminar "Implementing the Advanced Encryption Standard" and will be supplemented by an actual implementation and the documentation of this implementation.

\section{Organization of this Thesis}
\label{ch:organizationofthisthesis}

The second chapter adresses the background of AES. First a quick overview on the developements of digital encryption is given and what AES means in that context. Then the selection process of AES is discussed. Finally some reactions of the cryptographic community to AES are gathered and summarized.
The third chapter describes in detail, how the project was planned. It walks through the different phases of project developement and applies each phase to the present project, giving insights in how the final result was realized.
