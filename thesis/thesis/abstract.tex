% !TEX root = thesis.tex
\begin{abstract}
The usage of smartphones has become quite popular in the last decade. Every new smartphone model contains new hardware features, thus making people more attracted to benefit from these features. But the more features a smartphone contains, the more potential privacy and security issues arise from these features. Many of these features utilize the various built-in smartphone sensors like the accelerometer. Accelerometer data lead to the ability for installed applications to track the user's actual condition and activity. Previous laboratory studies have proved that hardware imperfections during the accelerometer manufacturing process, provide the possibility to recognize smartphones by utilizing signal feature extraction. In our approach, we demonstrate whether this method is applicable to recognize device models just as well as unique devices.
\end{abstract}
\clearpage