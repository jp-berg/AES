% Always start a chapter with a short but informative text about the following sections\@. Point out the relevance of the sections and create interconnections between them\@. Never ever just write a single sentence here\@. Furthermore,\ you are strongly advised to respect the hints given in this template\@.

In this chapter, we give an overview on our motivation and the related work done so far on fingerprinting and recognizing accelerometers, followed by our contribution to this field of studies. 

\section{Motivation}
\label{ch:motivation}
% What is your motivation to deal with this subject\@? Which interesting problems do you expect\@? Do not abbreviate \enquote{e.\,g.\/} within a sentence,\ always write \enquote{for example}\@. However,\ within in parentheses you are allowed to abbreviate and use,\ e.\,g.,\ and, i.\,e.,\ as shown here: with a comma right before and after it\@. In addition to that,\ ensure correct spacing by using \texttt{\textbackslash,} in between\@.
The ability to recognize devices such as tablets and desktops has been discussed quite often in the last decade. Methods such as supercookies and tracking static IP addresses, and tools like panopticlick~\cite{eckersley2010unique} have been utilized to recognize this kind of devices. With the growing popularity of mobile devices, advertising companies have also developed an interest in this category of devices. The challenge of recognizing mobile devices remains in the great similarity between the hardware and software specification of device models. Additionally, traditional tracking options have been restricted to prevent mobile devices being fingerprinted. For instance, in case of tracking cookies, the cookie law~\cite{CookieLaw} which regulates storing and retrieving information on a computer, smartphone, or tablet of visitors to a website in EU countries was passed on May 26 2011. User-defined settings in case of privacy-concerning features, e.g., GPS sensors prevent applications from tracking the current device location. Another example is Apple's improved policy since the presentation of iOS 7. Apple's iOS application developers are prohibited from using unique device features such as the \ac{UDID}, \ac{IMEI}, and \ac{MAC} address of the Wi-Fi interface as recognition mechanism. These effective preventive measures have caused adversaries to look for fingerprinting methods that do not require any sort of permission. Getting deeper into the modern features of mobile devices, it turns out that the accelerometer does not need any of the mentioned permissions, proved by various gaming and fitness applications that have never required permission to access the accelerometer readings so far. Gaming applications need accelerometer readings to track device movements and perform predefined actions, according to recognized patterns from the accelerometer readings. Fitness tracking applications use accelerometer readings to distinguish between user's physical activities and extract the resulting heart rate~\cite{kwon2011validation}.
Previous studies have illustrated that calibration imprecisions and hardware imperfections during the manufacturing process lead to unique hardware-tolerances. We refer to these studies in Section~\ref{ch:relatedwork}. By combining these unique imprecisions and lack of necessity of obtaining permission to read accelerometer readings, an adversary can calculate unique accelerometer fingerprints of a mobile device.

\section{Related Work}
\label{ch:relatedwork}
% List related work \emph{and} the result of this work\@! What is the relevance of this work concerning your thesis\@? If necessary,\ \emph{emphasize} some words in your text, for example words like \emph{not} or \emph{and} are sometimes crucial for understanding\@.
The power of extracting time and frequency domain features in accelerometers has been shown by W.~Dargie and M.~Denko~\cite{dargie2010analysis}. Their study includes random placement of accelerometers on moving humans and cars, and investigating the behavior of accelerometers during similar movements. They conclude that the extracted frequency domain features remain generally more robust than time-domain frequency features. In our study, we applied accelerometer readings that were gathered from both resting and moving devices. The features that were analyzed in this paper differ from the features that we used. Yet, several time domain features such as mean, standard deviation, and highest and lowest value, as well as frequency domain features such as spectral centroid are common in both works.\\
S.~Dey et al.~\cite{dey2014accelprint} created fingerprints for mobile devices by extracting time and frequency domain features from the accelerometer readings of vibrating sensors and mobile devices. Their study illustrates the existence of such unique fingerprints by conducting a series of training and test set scenarios on 107 different stand-alone chips, smartphones, and tablets under laboratory conditions. This paper is the main basis of this thesis. In this thesis, the duration of the time window of the accelerometer readings amounts 10 seconds while the AccelPrint paper used accelerometer readings with a duration of 2 seconds. A shorter time window leads to a more stationary signal which also provides more accurate feature extraction results and also more fingerprints for the training and test phase. Other than the applied bagging tree classifier in AccelPrint, we also tested random forests, extra-trees, and gradient boosting to have a better overall overview on device and model recognition. Additionally, we could compare the behavior of different ensemble methods.\\
Bojinov et al. presented accelerometer-based fingerprinting by applying JavaScript code to read out the accelerometer readings~\cite{bojinovmobile}. They attempted to recognize smartphones by calculating two bias parameters from the \textit{z} axis while the device was facing up or down on a resting surface. Their best and worst recognition rates were 8.3 and 15.1\%.\\
Application of the machine learning library \textit{scikit-learn} on recognizing human activities has been shown by H. He~\cite{he2013human}. This study used the accelerometer and gyroscope readings of 6 different human activities. The number of features that were extracted in this study were 561, which is much more than our 17 extracted features. In a PCA (principal component analysis) dimension reduction, He reduced the number of features to train the applied classifiers to 50 and 20. In the later number of features which is closer to the number of features used in our study, the achieved results in the random forests classifier are similar to our result (cf. Chapter~\ref{chap:results}). The other applied classification methods were different from our methods. We preferred to apply ensemble methods as classification methods and also test clustering methods for device and model recognition purposes.

\section{Contribution}
\label{ch:contribution}
Our contribution can be summarized as follows:
\begin{enumerate}
	\item We implemented the signal feature extraction process utilizing the \textit{NumPy} and \textit{SciPy} libraries and, where necessary, developed our own implementation of the formulas. This process was especially essential regarding the spectral feature extraction.
	\item We show that besides recognition of unique mobile devices, it is also possible to recognize the device model. For this purpose, we conducted two similar tests where the same minimum number of fingerprints was applied to have a precise comparison between the results.
	\item The comparison of 4 ensemble classification methods of the \textit{scikit-learn} library to recognize unique mobile devices and device models is also part of this thesis. We compare the results of the classification methods and discuss their efficiency on data sets that are either free of noise or contain fingerprints created from faulty accelerometer readings. 
	\item To the best of our knowledge, this is the first work where clustering is attempted to recognize mobile devices or the model of a device through their accelerometer fingerprints. We conducted tests with two different clustering methods of \textit{scikit-learn} to find out if clustering is also applicable to group accelerometer fingerprints correctly.
\end{enumerate}

\section{Organization of this Thesis}
\label{ch:organizationofthisthesis}
% Please give a general overview on how your thesis is divided into sections and chapters~\dots
The rest of this thesis is organized as follows. Chapter~\ref{chap:background} discusses the accelerometer briefly and introduces the time and frequency domain features and libraries that were used in this thesis. In Chapter~\ref{chap:implementation}, we provide the implementation details and explain the tested machine learning methods. Chapter~\ref{chap:results} covers the data analysis of the data set and the results of applying machine learning on the accelerometer fingerprints. Chapter~\ref{chap:conclusion} presents the conclusion and possible future work.
\clearpage