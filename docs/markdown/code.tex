\begin{lstlisting}
/*
 * Adds the roundkey to a block.
 */
inline void AddRoundKey(uint8_t * restrict bytes, 
                        const uint8_t * restrict keys)
{
        for(uint8_t i = 0; i < 16; i++) {
                bytes[i] ^= keys[i];
        }
}
\end{lstlisting}

\begin{lstlisting}
/*
 * Achieves the AES-ShiftRows by representing it as a series of array-assignments.
 */
void ShiftRows(uint8_t * restrict block, uint8_t * restrict tempblock)
{
        memcpy(tempblock, block, 16 * sizeof(uint8_t));

        block[1] = tempblock[5];
        block[2] = tempblock[10];
        block[3] = tempblock[15];
        block[5] = tempblock[9];
        block[6] = tempblock[14];
        block[7] = tempblock[3];
        block[9] = tempblock[13];
        block[10] = tempblock[2];
        block[11] = tempblock[7];
        block[13] = tempblock[1];
        block[14] = tempblock[6];
        block[15] = tempblock[11];
}
\end{lstlisting}

\begin{lstlisting}
@pytest.mark.parametrize(
    ("input_block", "expected"),
    (
        ("63cab7040953d051cd60e0e7ba70e18c", "6353e08c0960e104cd70b751bacad0e7"),
        ("a761ca9b97be8b45d8ad1a611fc97369", "a7be1a6997ad739bd8c9ca451f618b61"),
        ("3b59cb73fcd90ee05774222dc067fb68", "3bd92268fc74fb735767cbe0c0590e2d")
    )
)
def test_ShiftRows(input_block, expected):
    """Tests the C-implementation of the ShiftRows-function"""
    ba = bytearray.fromhex(input_block)
    reference = bytearray.fromhex(expected)
    byte_array = ctypes.c_ubyte * len(ba)
    temp_array = ctypes.c_ubyte * len(ba)
    aeslib.ShiftRows(byte_array.from_buffer(ba), temp_array.from_buffer_copy(ba))
    assert ba == reference
\end{lstlisting}

\begin{lstlisting}
/*
 * Achieves the AES-MixColumns by representing each result byte as a series of
 * table-lookups XORed with each other.
 */
void MixColumns(uint8_t * restrict block, uint8_t * restrict tempblock, 
                const uint8_t (* restrict gal_mult_lookup)[256])
{
        memcpy(tempblock, block, 16 * sizeof(uint8_t));

        for(uint8_t i = 0; i < 16; i += 4) {

                block[i] = gal_mult_lookup[1][tempblock[i]] ^
                        gal_mult_lookup[2][tempblock[i+1]] ^
                        gal_mult_lookup[0][tempblock[i+2]] ^
                        gal_mult_lookup[0][tempblock[i+3]];

                block[i+1] = gal_mult_lookup[0][tempblock[i]] ^
                        gal_mult_lookup[1][tempblock[i+1]] ^
                        gal_mult_lookup[2][tempblock[i+2]] ^
                        gal_mult_lookup[0][tempblock[i+3]];

                block[i+2] = gal_mult_lookup[0][tempblock[i]] ^
                        gal_mult_lookup[0][tempblock[i+1]] ^
                        gal_mult_lookup[1][tempblock[i+2]] ^
                        gal_mult_lookup[2][tempblock[i+3]];

                block[i+3] = gal_mult_lookup[2][tempblock[i]] ^
                        gal_mult_lookup[0][tempblock[i+1]] ^
                        gal_mult_lookup[0][tempblock[i+2]] ^
                        gal_mult_lookup[1][tempblock[i+3]];
        }
}
\end{lstlisting}

\begin{lstlisting}
<# Vectors from [FIPS 197] Appendix C, Rounds 1, 2, 3
@pytest.mark.parametrize(
    ("input_block", "expected"),
    (
        ("6353e08c0960e104cd70b751bacad0e7", "5f72641557f5bc92f7be3b291db9f91a"),
        ("a7be1a6997ad739bd8c9ca451f618b61", "ff87968431d86a51645151fa773ad009"),
        ("3bd92268fc74fb735767cbe0c0590e2d", "4c9c1e66f771f0762c3f868e534df256")
    )
)
def test_MixColumns(input_block, expected):
    """Tests the C-implementation of the MixColumns-function"""
    ba = bytearray.fromhex(input_block)
    reference = bytearray.fromhex(expected)
    byte_array = ctypes.c_ubyte * len(ba)
    temp_array = ctypes.c_ubyte * len(ba)
    gal_mult_lookup_array = ctypes.c_ubyte * len(gal_mult_lookup)
    aeslib.MixColumns(byte_array.from_buffer(ba), temp_array.from_buffer_copy(ba),
                      gal_mult_lookup_array.from_buffer(gal_mult_lookup))
    assert ba == reference
\end{lstlisting}

\begin{lstlisting}
/*
 * Substitutes all bytes in a block with bytes from a sbox.
 */
inline void SubBytes(uint8_t * restrict bytes, const uint8_t * restrict sbox)
{
        for(uint8_t i = 0; i < 16; i++) {
                bytes[i] = sbox[bytes[i]];
        }
    
}
\end{lstlisting}

\begin{lstlisting}
void encryptBlock(uint8_t * restrict block, uint8_t * restrict tempblock, 
                  const uint8_t * restrict keys, const uint8_t rounds, 
                  const uint8_t * restrict sbox, 
                  const uint8_t (* restrict gal_mult_lookup)[256])
{   
        uint8_t ikeys = 0;
        AddRoundKey(block, keys);
        ikeys += 16;

        for(uint8_t i = 0; i < rounds - 1; i++) {   
                SubBytes(block, sbox);
                ShiftRows(block, tempblock);
                MixColumns(block, tempblock, gal_mult_lookup);
                AddRoundKey(block, &keys[ikeys]);
                ikeys += 16;
        }
        
        SubBytes(block, sbox);
        ShiftRows(block, tempblock);
        AddRoundKey(block, &keys[ikeys]);
}
\end{lstlisting}

\begin{lstlisting}
@pytest.mark.parametrize(
    ("plaintext", "key", "expected"),
    (
        (
            "3243f6a8885a308d313198a2e0370734",
            "2b7e151628aed2a6abf7158809cf4f3c",
            "3925841d02dc09fbdc118597196a0b32"
        ),
    )
)
def test_EncryptBlock(plaintext, key, expected):
    """Tests the C-implementation of the AES-Encryption of a single block"""
    ba = bytearray.fromhex(plaintext)
    byte_array_block = ctypes.c_ubyte * len(ba)

    reference = bytearray.fromhex(expected)
    baKey = expand_key(key)
    byte_array_key = ctypes.c_ubyte * len(baKey)

    temp_array = ctypes.c_ubyte * len(ba)
    sbox_array = ctypes.c_ubyte * len(sbox)
    gal_mult_lookup_array = ctypes.c_ubyte * len(gal_mult_lookup)
    

    aeslib.encryptBlock(byte_array_block.from_buffer(ba), temp_array.from_buffer_copy(ba),
                        byte_array_key.from_buffer(baKey), 10, sbox_array.from_buffer(sbox),
                        gal_mult_lookup_array.from_buffer(gal_mult_lookup))
    assert ba == reference
\end{lstlisting}


\begin{lstlisting}
def test_EncryptBlockRandom():
    """
    Tests the C-implementation of the AES-Encryption of an arbitrary number of individual blocks.
    Plaintext and Key are randomized each time.
    """
    ba = bytearray(16)
    byte_array_block = ctypes.c_ubyte * len(ba)
    key = bytearray(16)
    baKey = bytearray(176)
    byte_array_key = ctypes.c_ubyte * len(baKey)
    temp_array = ctypes.c_ubyte * len(ba)
    sbox_array = ctypes.c_ubyte * len(sbox)
    gal_mult_lookup_array = ctypes.c_ubyte * len(gal_mult_lookup)
    for i in range(100): #arbitrary number of rounds
        key = urandom(16)
        baKey = expand_key(key.hex())
        b = urandom(16)
        ba = bytearray(b)
        aes_reference = pyaes.AESModeOfOperationECB(key)
        reference = aes_reference.encrypt(b)
        aeslib.encryptBlock(byte_array_block.from_buffer(ba), temp_array.from_buffer_copy(ba),
                        byte_array_key.from_buffer(baKey), 10, sbox_array.from_buffer(sbox),
                        gal_mult_lookup_array.from_buffer(gal_mult_lookup))
        assert ba == reference
\end{lstlisting}

\begin{lstlisting}
/*
 * Initializes the constant tables sbox, galois-field-multiplication-lookup and keys.
 * Performs AES-encryption on multiple, consecutive blocks.
 */
void encryptAES(uint8_t * restrict bytes, uint8_t * restrict initval, 
                   const size_t bytecount, const uint8_t rounds)
{   
        uint8_t sbox[256];
        uint8_t gal_mult_lookup[3][256];
        for(size_t i = 0; i < 256; i++) {
                sbox[i] = *initval++;
        }
        for(size_t i = 0; i < 3; i++) {
                for(size_t j = 0; j < 256; j++) {
                gal_mult_lookup[i][j] = *initval++;
                }
        }
        
        const uint8_t *keys = initval;
        uint8_t tempblock[16];
        
        for(size_t i = 0; i < bytecount; i += 16) {
                encryptBlock(&bytes[i], tempblock, keys, 
                             rounds, sbox, gal_mult_lookup);
        }
}
\end{lstlisting}

\begin{lstlisting}
def test_encrypt_aes():
    """Tests the C-implementation of the AES-Encryption on multiple, consecutive, random Blocks.
    """
    no_bytes = 2**15 #arbitrary number of bytes (has to be divisible by 16)
    ba = bytearray(no_bytes)
    byte_array_block = ctypes.c_ubyte * len(ba)
    key = bytearray(16)
    baKey = bytearray(176)
    initvals = bytearray(len(sbox) + len(gal_mult_lookup) + len(baKey))
    byte_array_initvals = ctypes.c_ubyte * len(initvals)
    for i in range(20): #arbitrary number of rounds
        key = urandom(16)
        baKey = expand_key(key.hex())
        initvals = sbox + gal_mult_lookup + baKey
        b = urandom(no_bytes)
        ba = bytearray(b)
        aes_reference = pyaes.AESModeOfOperationECB(key)
        aeslib.encryptAES(byte_array_block.from_buffer(ba),
                        byte_array_initvals.from_buffer(initvals), len(ba), 10)

        i = 0
        while (i < no_bytes):
            current_reference = aes_reference.encrypt(b[i:i+16])
            current_ba = ba[i:i+16]
            assert current_ba == current_reference

            i += 16
\end{lstlisting}
