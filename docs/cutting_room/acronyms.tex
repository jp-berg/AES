%!TEX root = thesis.tex

% List of acronyms
%-- \ac{betai} f�gt die Abk�rzung ein. Beim ersten Vorkommen der Abk�rzung wird zun�chst der ganze Begriff angegeben und das Akronym bzw. die Abk�rzung in Klammern.
%-- \acs{betai} wird dann nur das Symbol bzw. die Abk�rzung im Text gedruckt
%-- \acf{betai} gibt zus�tzlich noch die Erkl�rung aus
%-- \acl{betai} f�gt lediglich die Beschreibung ein.
%\chapter{Acronyms}
\begin{acronym}
 \setlength{\itemsep}{0.2em}
 \acro{BPMN}{business process model and notation}
 \acro{FFT}{fast Fourier transformation}
 \acro{IMEI}{international mobile station equipment identity}
 \acro{MAC}{media access control}
 \acro{MEMS}{microelectromechanical system}
 \acro{PCA}{principal component analysis}
 \acro{RSS}{root sum square}
 \acro{UDID}{unique device identifier}
 \acro{CPU}{central processing unit}
 \acro{GFMLT}{Galois field multiplication lookup table}
 \acro{$GF(2^{8})$}{Galois field in $2^{8}$}
 \acro{ECB}{Electronic Codebook Mode}
 \acro{I/O}{input/output}
 \acro{x86}{x86 instruction set family}
 \acro{XOR}{bitwise exclusive or operation}
 \acro{AND}{bitwise and operation}
 % etc.
\end{acronym}
