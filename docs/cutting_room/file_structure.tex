%!TEX root = thesis.tex
\section{File Structure}
The main file of the program is \lstinline|wrapper.py|. It contains the command line interface, all the setup logic, as well as the input and output handling. All non-standard python dependencies are listed within the requirements file. This way they can all be installed with one pip command. The README contains an overview over the program, as well as usage and installation instructions.


\begin{lstlisting}
wrapper.py                                    lib/
aes                                             libaes_encrypt.so
src/                                            libaes_decrypt.so
  AES_encrypt.h                               test/
  AES_encrypt.c                                 __init__.py
  AES_decrypt.h                                 test_AES_decrypt.py
  AES_decrypt.c                                 test_AES_encrypt.py
  key_expansion.py                              test_key_expansion.py
  AES_generator.py
README.md
requirements.txt
\end{lstlisting}


The source directory contains all of the AES algorithm functionality: en- and decryption of blocks, key expansion, and S-Box generation. It is structured into the source code and header files for the en- and decryption code, one python file for the key expansion, and one for the generation of lookup tables (generator). For python the directory constitutes a module that can be imported in the wrapper and testing files. All functions of the en- and decryption as well as key expansion files are explained in the appropriate sections of the documentation.
