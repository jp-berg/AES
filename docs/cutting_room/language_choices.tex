%!TEX root = thesis.tex
\hypertarget{language-choices}{%
\section{Language Choices}\label{language-choices}}

We chose to implement the algorithm itself in C, while Python handles
the input and output of our program. This allows us to use both
languages to their full potential, while avoiding problems of both.

\hypertarget{python}{%
\subsection{Python}\label{python}}

Python is a popular (\cite{instack}, \cite{octgit}) multi-paradigm (\cite[p. 6]{learningpython})
programming language. According to \cite{wikipython} it is interpreted,
high-level and general purpose. Combining its high usage with its thirty
years of age it can be considered a mature language and as such boasts a
fair share of success stories (\cite{pythonsuccess}, \cite[p. 9 - 10]{learningpython}. With
over 280000 Packages available via the Python package manager pip it
offers a healthy and extensive ecosystem (\cite{pythonpackages}).

With our design we are able to make use of Pythons advantages:

\begin{itemize}

\item
  \emph{Readability}: One of Pythons qualities is its ``focus on
  readability, coherence, and software quality in general''
  (\cite[p. 3]{learningpython}). C on the other hand is generally considered to be
  more difficult to understand, especially because of features and
  concepts not found in other languages (\cite[p. 5]{cmodern}). This implementation
  therefore tries to cover as much of the functionality with the easier
  to read Python.
\item
  \emph{Memory safety}: All memory used by our program is allocated by
  python. Since it is a garbage-collected language, the program does not
  have to worry about freeing unused memory. This avoids memory leaks in
  the program. (\cite[p. 18]{learningpython})
\item
  \emph{Performance}: Python is on the slow end of programming
  languages, but that is nearly irrelevant for the purpose it is used in
  the application. The main task is handling input/output and on modern
  computers it should be fast enough to read/write from/to the command
  line or disk, without becoming a bottleneck.
\item
  \emph{Robustness}: The programming language itself is considered
  robust and stable  input in Python is relatively
  simple and straightforward (\cite[p. 9]{learningpython}). There is no undefined behavior, input
  length or encoding issues the programmer has to worry about (compare
  to \cite{scanf}).
\item
  \emph{Extensibility}: Pythons package manager, import system and large
  standard library makes it easy to add new features to our program.
  Adding password hashing for example could be accomplished by simply
  importing a new package from the standard library and adding a few
  lines of code. This opens up a lot of possibilities for extending the
  functionality of our program in the future. Here Python helps the
  present implementation to cover up for the lack of a package manager
  and a less comprehensive standard library in C(compare
  \cite{clib} to \cite{pybib}).
\item
  \emph{Productivity}: Python is a dynamically typed (\cite[p. 9]{learningpython}), interpreted language, generating a tight development loop
  uninterrupted by compile times, with almost no delay between writing
  and executing the code. This helps to improve and iterate over the
  code rapidly, motivating a more exploratory and incremental approach
  to program development. Furthermore Python is able to express more
  functionality in less lines of code than other languages, which helps
  with readability and maintainability. ``It is deliberately optimized
  for speed of development'' (\cite[p. 5]{learningpython})
\item
  \emph{Portability}: Most Python code is able to run unmodified on
  ``all major computer platforms''(\cite[p. 4, 17-18]{learningpython}). Combined
  with the portability of C our program is only limited by its ability
  to execute the compilation of the C-core, which for now is bound by
  the ability of Python dispatching the compile command to GCC or Clang,
  which for now was only tested on Ubuntu, but should theoretically work
  on at least all Linux Systems with Python and Clang or GCC installed.
\end{itemize}

\hypertarget{c}{%
\subsection{C}\label{c}}

C is, like Python, one of the most popular programming languages today
(\cite{instack}, \cite{octgit}). It is close to 20 years older than Python though,
dating back to the early 1970s (\cite[p. 2]{cmodern}). It is statically typed
(\cite[p. 40]{modernc}), compiled (\cite[p. 2]{cmodern}), low-level (meaning providing
access to concepts more commonly found on the machine level) (\cite[p. 4]{cmodern}) and provides interfaces for manual memory management, but no build in
garbage collection (\cite[p. 55]{cpointers}). One of the more notable projects
written in C is the Linux kernel, the basis for the widely used Linux
operating system (\cite{kernel}).

C is considered one of the fastest and most efficient languages
(\cite{benchmarkgame}), but is
more error-prone than others (\cite[p. 5]{modernc}). One has to be careful to avoid
so called ``undefined behaviour'', buffer overflows (since C-arrays are
not bounds-checked), and memory management bugs.

A popular example of a serious bug that gets attributed to the fact that
C is more error-prone than other languages is Heartbleed. A programming
mistake so serious, that ``[s]ome might argue that it is the worst
vulnerability found (at least in terms of its potential impact) since
commercial traffic began to flow on the Internet.'' (\cite{forbes}).
\cite{wheeler} sees this as a
direct result of C not including ``any built-in detection or
countermeasure for improper restriction of buffers'' (ch. 3.9) and
states that one of ``the most dangerous widely-used languages for
security-relevant software [is] C''. This type of vulnerability is
``widely used'' often with ``catastrophic effect'' , but ``Using or
switching to almost any other language (other than C, C++, or
Objective-C) would completely eliminate buffer-related vulnerabilities,
including Heartbleed.'' (ch. 3.9.1)

The following list illustrates how the program uses the advantages of C,
while avoiding the shortcomings of the language:

\begin{itemize}

\item
  \emph{Speed}: Since C is considered a fast language (\cite[p. 4]{cmodern}) it is
  not unusual to rewrite calculation intensive parts of programs in C,
  even if the project is written in another language
  (\cite[min. 24:33]{speedup}). Python itself
  uses this architectural concept, interfacing with compiled C code to
  make use of Cs speed advantage for certain tasks, like \cite[p. 7]{learningpython} explains. The author even recommends to create compiled extensions
  for ``domains that do require optimal execution speeds'' and linking
  those back into the Python code. The calculation intensive part of our
  project is clearly the AES encryption itself, especially with growing
  file size. Additionally the architecture of our program allows the C-core to take advantage of '-march=native',
  a compiler-flag that enables tailoring the binary to the features of the underlying CPU,
  resulting in even more performance. And this happens automatically without any input required from the user.
\item
  \emph{no memory management necessary}: Due to the fact that the C-core
  operates on byte-arrays allocated by Python, nothing needs to be
  allocated on heap. Since the rest of the data needed is allocated on
  the stack all memory used is automatically freed after the encryption
  function returns, without the need for the programmer to intervene.
  This avoids memory leaks occurring when allocations on heap are not
  met with a call to free() after the allocated memory is not needed
  anymore. So called double free errors are also made impossible (since
  there is nothing to call `free()' on), meaning there is no risk of
  freeing the same area of memory twice. This error could potentially
  corrupt the programs memory management data structures and lead to a
  crash(\cite[p. 49]{cpointers}).
\item
  \emph{straightforward implementation}: The algorithm described in
  \cite{aes2001} translates well to C code. No special imports are needed, no
  hard-to-understand operations are required to be executed and the
  depicetd pseudocode functions map reasonably cleanly to C functions.
  The AES-algorithm itself makes extensive use of
  Galois-Field-operations in $2^8$. Those can be easily expressed
  through the native uint8\_t C-data-type, which represents an unsigned
  byte. Buffer overruns are avoided, because the array AES operates on
  is allocated by Python. Since Python tracks the length of any object
  allocated and since we pass the length of the array as a boundary,
  buffer overruns are highly unlikely. This is one example of how the
  wrapper ensures the data has the right format, leading to less error
  handling and thus keeping the C-core simple.
\item
  \emph{fast compile times}: Since the C-core encompasses only a few
  hundred lines of code it compiles in subsecond time on our development
  machines, even with optimization enabled. This encourages trying new
  things and helps to get quick feedback even for small changes, either
  through the static analyzer of the compiler itself or through the
  included testing libraries. The small and thus fast-compiling C-core
  is only possible by outsourcing all less calculation intensive tasks to
  the wrapper.
\item
  \emph{portability}: C compiles to numerous platforms and operating
  systems. \cite{gccsys} demonstrates how
  many architectures and operating systems can be reached with just one
  compiler. Since the C-core tries to avoid any platform specifics or
  targeting specific compilers, it should not be the limiting factor in
  terms of portability, especially since the reference implementation of
  Python is written partially in C (\cite{pyref}),
  which means Python is not able to target CPU architectures or
  operating systems C cannot run on.
\end{itemize}
