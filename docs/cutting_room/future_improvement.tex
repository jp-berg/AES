%!TEX root = thesis.tex
\hypertarget{possible-areas-of-future-improvements}{%
\chapter{Possible areas of future
improvements}\label{possible-areas-of-future-improvements}}

This chapter is a collection of ideas, which could bring potential
improvements to the project in the future.

\hypertarget{aes-192-and-aes-256}{%
\section{AES-192 and AES-256}\label{aes-192-and-aes-256}}

Currently the implementation only supports AES with key sizes of 128
bits, but it would be possible to also support other lengths of the
standard, namely key lengths of 192 and 256 bits. Changes to the AES
code itself would be minimal since the only difference would be the
number of rounds, which is already a flexible parameter in the
encryption implementation. The prep\_password-function has to be
tweaked, but the used hashing function makes adjusting the key length
easy. The test would have to be modified too, although this modification
seems more like a hassle than a challenge since the logic, that is
already in place for AES-128, only needs to be adjusted for the
additional key sizes. Even modifying the key expansion to accommodate
for two more sizes should be easily doable.

\hypertarget{parallelism}{%
\section{Parallelism}\label{parallelism}}

Since AES encrypts blocks independently from each other the algorithm
can be categorized as an so called embarrassingly parallel problem
\cite[p.48]{parallelprog}. Consequently an easy avenue for gaining more
speed seems to be restructuring the program in such a way, that every
available CPU-core gets put to work on a separate series of blocks to
encrypt/decrypt. On short encryption operations in the kilobyte range
this may not deliver a considerable performance increase, but since the
present implementation is able to operate on files of unbounded size,
time saving possibilities become apparent quite fast.

A simple way to archive such parallelism could for example lead through
the OpenMP-library. This library was designed to provide a shared-memory
API that allows the program to execute in parallel and is added into the
program fairly easily (\cite[ch. 5]{parallelprog}. Since the encryption of
multiple blocks can be expressed as a for-loop, OpenMPs `parallel
for'-directive may be an easy way to execute the program on multiple
cores (compare (\cite[ch. 5.5]{parallelprog}).

\hypertarget{t-tables}{%
\section{T-tables}\label{t-tables}}

The present implementation already makes use of lookup tables. \cite[p. 59]{rijndael} proposes a method that makes even more extensive use of lookup
tables, which needs only 4 array accesses and 4 bitwise XOR operations
for each round and 4-byte column. They budget in 4kb of memory for the
lookup tables. This is a negligible amount of memory compared to the
gigabytes available on modern x86-systems.

\hypertarget{no-constant-value-calculations}{%
\section{No constant value
calculations}\label{no-constant-value-calculations}}

To fulfill the task of this seminar it was explicitly required to (re-)
calculate values, that could as well be hardcoded. The additional space
used by this is calculated as follows: The GFMLT for multiplication with
two and three can be stored in 2 * 256 bytes (multiplication with one
does not need to be stored, since it just results in the unmodified
value itself). The S-box and Inverse S-box are 256 additional bytes
each, which brings us to a total of 1024 additional bytes of memory.
This is considered to be a negligible size increase, but saves on
recalculating the values on program startup.

\hypertarget{block-cipher-mode-of-operation}{%
\section{Block cipher mode of
operation}\label{block-cipher-mode-of-operation}}

Due to time constraints, it was not possible to implement a different
block cipher mode of operation, but it is something the current
implementation would greatly benefit from. \ref{encryption-of-multiple-consecutive-blocks} goes more
in-depth regarding this topic.

\hypertarget{file-header-argon2-hashing-algorithm-salt-and-hmac}{%
\section{File-header: ARGON2-Hashing algorithm, salt and
HMAC}\label{file-header-argon2-hashing-algorithm-salt-and-hmac}}

At the moment the passwords are hashed with PBKDF2. This algorithm can
unfortunately easily be cracked with the right hardware, like \cite[p. 697]{appcrypt} explains and suggests (on p. 701) using the password hashing
competition winner argon2, which is designed to withstand brute force
attempts by being able to use more memory and parallelism, thus making it
more expensive for an attacker to guess the password. The
argon2-parameters would either have to be hardcoded or attached to the
encrypted text in a header for example.

If such a header would be added, it seems like a logical step to also
add a public salt to the hash parameters specified within. A salt is a
random string, regenerated for each new password, that is hashed along
with the password to prevent any kind of preprocessing attack on the
password hash. The salt itself is stored in the clear. \cite[p. 693]{appcrypt} At
the moment the hash is hardcoded to ``aeskurs'', but in the future would
consist of a randomly generated string from ``os.urandom()''.

The header could also hold the hash for the hash based message
authentication code or short HMAC (\cite[ch. 12.2.3]{paar}). This would ensure
integrity of the message, i.e.~that the encrypted data was not changed,
for example during transit by a malicious third party. This is archived
by hashing the message before transmission in combination with a secret
shared only by the communicating parties (like the encryption-password)
and appending the resulting hash to the message. The receiving party is
then able to verify the integrity of the message by repeating the steps
and comparing the generated and the appended hash.

\cite{moxie} recommends to authenticate first before performing any
cryptographic operation, so a future implementation would probably first
encrypt the message, then hash it and append the hash to the header.
Since we want to ensure that the correct password-hashing parameters
were received, those would be also fed into the HMAC if a future
implementation makes use of said parameters. The HMAC algorithm would
probably a SHA3-variant, as specified in \cite{fips202}, providing the HMAC
with adequate cryptographic strength.
