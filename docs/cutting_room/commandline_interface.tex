%!TEX root = thesis.tex
\section{Commandline Interface}
The only way to interact with the program is via the commandline. In order to provide a usable interface for the user, the python module "click" is used. The module provides ways to pass command line arguments to function without a lot of code. This is achieved via the python concept knows as decorators.

Click or \enquote{Command Line Interface Creation Kit} is a python package for creating command line interfaces \cite{click}. The simplicity of the packages comes from how the cli is constructed. By adding a \lstinline{@click.command()} decorator to a function, the function becomes directly callable from the command line. Further decorators add options, defaults, callbacks, and arguments. To combine multiple commands for a program a concept known as \enquote{groups} is used.

Click automatically generates help messages for the user. The basis for these messages are the docstrings found in the code, as well as the arguments of the functions. They either appear, when the program is run without a command (listing available commands) or when a command is run without arguments (providing documentation for the function).
