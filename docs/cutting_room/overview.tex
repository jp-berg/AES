%!TEX root = thesis.tex
\hypertarget{overview}{%
\chapter{Overview}\label{overview}}

This is the documentation to an implementation of the Advanced Encryption Standard (AES). It outlines the general architecture, as well as how specific functions are implemented.

The C programming language is used for implementing
AES. The source code for the algorithm is compiled to a ``Shared
Object'', which is then loaded as a ``dynamically loaded library'' into
a Python wrapper, also referred to as wrapper, which facilitates the
interaction with the algorithm. More information on the wrapper can be found in section~\ref{the-wrapper}.

The use of two programming languages allows us to leverage the advantages
of both, while alleviating some shortcomings that they possess. The
python wrapper provides a command line interface, which enables
encryption and decryption of files and text input. The c source code
describing the implementation of the AES-algorithm, also referred to as
C-core, has test coverage for every function to ensure correctness.

On calling the wrapper via command line, the program checks
whether there already are .so-files available to link. If the wrapper
cannot find them it compiles them from the .c-files. After that it links
the libraries and executes the command. It transforms the passed user
input, either file or text, into a padded byte-array, and generates the
key. There is either a 16-byte hex key made available by the user or it
has to be generated via a hash function from a string input by the user.
After the key expansion (in Python) both the key and the byte-array with
the user input get passed on to the C-core. The core gets a pointer to
the array and key. It uses those pointers to transform the
user-input-array in place. After the transformation is finished the
wrapper just needs to output the transformed byte-array.

Since the only differences between AES and Rijndael are the sizes of the
accepted keys and blocks (\cite[p. 31]{rijndael}) the names for both algorithms
will be used synonymously throughout this document.
